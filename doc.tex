\documentclass[12pt, titlepage, a4paper, oneside]{article}

\usepackage{graphicx}
\usepackage{verbatim}

\setlength{\oddsidemargin}{0in} \setlength{\evensidemargin}{0in}
\setlength{\textwidth}{6.2in}
\setlength{\topmargin}{-0.3in} \setlength{\textheight}{9.8in}

\renewcommand{\familydefault}{\sfdefault}

\title{Word Play}
\author{Tom Leaman}

\begin{document}
\maketitle

\section*{Design}

\subsection*{UML Diagram}
%\begin{center}
%\includegraphics[scale=0.70]{useCase.png}
%\end{center}

\subsection*{Design Justification}
I have tried, where possible, to declare my variables using the type of an
Interface rather than a concrete type. This improves the loose coupling of
various classes.
With that in mind, I also developed the graph data structure as an Interface,
using generic type definitions. This means that, should the implementation of
the graph change at a later date, the code should not break entirely.

\subsubsection*{Generator}
I decided to implement the Generation task recursively, to produce a depth
first search. I decided to make use of a LinkedList for the ladder itself as
nodes were going to be added and removed a lot as the program searched the graph.
I used a HashSet to store nodes that had already been visited. Since this structure
is really only used to store entries and then do look-ups, it should benefit
from better performance (especially on large data sets) due to its use of
a hashing function.

\subsubsection*{Discoverer}
I decided to try to use a simplified version of Dijkstra's algorithm to find
the shortest path between two nodes. I found a much more complete version of
the algorithm in Java (link in source) and modified its design for this task.
I found this task incredibly challenging and so I thought a little less long
and hard about the data structues I was using here. I have still maintained
the principal of coding to an Interface rather than a class, however.

\section*{Algorithm Explaination}

\subsection*{Building the graph}
I kept the design of the graph structure as simple as possible, it simply maps
a String to a List<String> which are pointers to other nodes in the graph. These
are calculated when the graph is first built by reading in a file and then checking
each word as it's entered against all the other words in the graph. Strings which
are exactly one character different will be added to each other's List.

\subsection*{Generator}
The generateLadder function provides a wrapper that handles the recursive method
and cleans up the failure state a little before returning.

Here is the pseudo code for the recursive function:
\begin{verbatim}
  def search( ladder, toFind ):
    if toFind == 0 return;

    Node current = ladder.getLast()
    List<Node> next = graph.getConnected( current )
    next.remove( ladder, visited )
    // now we try every node we can reach
    for nextWord in next:
      ladder.add( nextWord )
      visited.add( nextWord )
      search( ladder, toFind - 1 )
      if ( ladder.length() == requiredDepth )
        return ladder
      // unsuccessful so we remove whatever we just tried
      ladder.removeLast() 
\end{verbatim}

\subsection*{Discoverer}
This algorithm works in 2 stages, first it traverses the graph and builds a map
that points from a node to its lowest cost parent node.
\begin{verbatim}
  Graph g
  List unvisited, ladder
  Map costs, parents

  costs.put( firstWord, 0 )
  unvisited.add( firstWord )
  while ( unvisited !empty ):
    String current = getLowestCostNode( unvisited )
    unvisited.remove(current)
    // update / add values for the nodes we can reach from current
    // in parents and costs
    updateCosts( graph.getConnected( current ))
    updateParents( graph.getConnected( current ))
\end{verbatim}

\par
The program then simply reads backwards through the map, starting with the target
word and storing each subsequent parent into the ladder until no parent can be
found. This indicates that we've found the start node and can return a
reversed list.
\begin{verbatim}
  String current = end
  if ( !hasParents( current )) return emptyLadder // no path can be found
  ladder.add( current )
  while ( hasParents( current )):
    current = parents.get( current ) // get the lowest cost parent
    ladder.add( current ) // add it to the ladder
  return reverse( ladder )
\end{verbatim}

\newpage
\section*{Source Code}

\subsection*{Driver.java}
\verbatiminput{src/ui/Driver.java}
\newpage

\subsection*{Graph.java}
\verbatiminput{src/graph/Graph.java}
\newpage

\subsection*{AbstractGraph.java}
\verbatiminput{src/graph/AbstractGraph.java}
\newpage

\subsection*{WordGraph.java}
\verbatiminput{src/graph/WordGraph.java}
\newpage

\subsection*{Generator.java}
\verbatiminput{src/ladder/Generator.java}
\newpage

\subsection*{Discoverer.java}
\verbatiminput{src/ladder/Discoverer.java}
\newpage

\subsection*{GraphTest.java}
\verbatiminput{src/test/GraphTest.java}
\newpage

\subsection*{GeneratorTest.java}
\verbatiminput{src/test/GeneratorTest.java}
\newpage

\subsection*{DiscovererTest.java}
\verbatiminput{src/test/DiscovererTest.java}
\newpage

\subsection*{Utils.java}
\verbatiminput{src/test/Utils.java}
\newpage

\section*{Example Output}

\subsection*{Generator}
\verbatiminput{gtest1}
\par
\verbatiminput{gtest2}
\par
\verbatiminput{gtest3}

\subsection*{Discoverer}
\verbatiminput{dtest1}
\par
\verbatiminput{dtest2}
\par
\verbatiminput{dtest3}

\end{document}
