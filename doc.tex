\documentclass[12pt, titlepage, a4paper, twoside]{article}

\usepackage{graphicx}
\usepackage{verbatim}

\setlength{\oddsidemargin}{0in} \setlength{\evensidemargin}{0in}
\setlength{\textwidth}{6.2in}
\setlength{\topmargin}{-0.3in} \setlength{\textheight}{9.8in}

\renewcommand{\familydefault}{\sfdefault}

\title{Word Play}
\author{Tom Leaman}

\begin{document}
\maketitle

\section*{Design}

\subsection*{UML Diagram}
%\begin{center}
%\includegraphics[scale=0.70]{useCase.png}
%\end{center}

\subsection*{Design Justification}
I have tried, where possible, to declare my variables using the type of an
Interface rather than a concrete type. This improves the loose coupling of
various classes.
With that in mind, I also developed the graph data structure as an Interface,
using generic type definitions. This means that, should the implementation of
the graph change at a later date, the code should not break entirely.

\subsubsection*{Generator}
I decided to implement the Generation task recursively, to produce a depth
first search. I decided to make use of a LinkedList for the ladder itself as
nodes were going to be added and removed a lot as the program searched the graph.
I used a HashSet to store nodes that had already been visited. Since this structure
is really only used to store entries and then do look-ups, it should benefit
from better performance (especially on large data sets) due to its use of
a hashing function.

\subsubsection*{Discoverer}
I decided to try to use a simplified version of Dijkstra's algorithm to find
the shortest path between two nodes. I found a much more complete version of
the algorithm in Java (link in source) and modified its design for this task.
I found this task incredibly challenging and so I thought a little less long
and hard about the data structues I was using here. I have still maintained
the principal of coding to an Interface rather than a class, however.

\section*{Algorithm Explaination}

\subsection*{Building the graph}
I kept the design of the graph structure as simple as possible, it simply maps
a String to a List<String> which are pointers to other nodes in the graph. These
are calculated when the graph is first built by reading in a file and then checking
each word as it's entered against all the other words in the graph. Strings which
are exactly one character different will be added to each other's List.

\subsection*{Generator}

\subsection*{Discoverer}

\newpage
\section*{Source Code}

\subsection*{Driver.java}
\verbatiminput{src/ui/Driver.java}
\newpage

\subsection*{Graph.java}
\verbatiminput{src/graph/Graph.java}
\newpage

\subsection*{AbstractGraph.java}
\verbatiminput{src/graph/AbstractGraph.java}
\newpage

\subsection*{WordGraph.java}
\verbatiminput{src/graph/WordGraph.java}
\newpage

\subsection*{Generator.java}
\verbatiminput{src/ladder/Generator.java}
\newpage

\subsection*{Discoverer.java}
\verbatiminput{src/ladder/Discoverer.java}
\newpage

\subsection*{GraphTest.java}
\verbatiminput{src/test/GraphTest.java}
\newpage

\subsection*{GeneratorTest.java}
\verbatiminput{src/test/GeneratorTest.java}
\newpage

\subsection*{DiscovererTest.java}
\verbatiminput{src/test/DiscovererTest.java}
\newpage

\subsection*{Utils.java}
\verbatiminput{src/test/Utils.java}
\newpage

\section*{Example Output}

% three examples for each mode

\end{document}
